\documentclass[10pt]{article}

\usepackage{times}
\usepackage[utf8]{inputenc}
\usepackage[french]{babel}
\usepackage{amsmath}
\usepackage{amsfonts}
\usepackage{amssymb}
\usepackage{graphicx}
\usepackage{parskip}
\usepackage{multicol}
\usepackage{url}

\begin{document}


\title{IFT2245 - Rapport TP1}
\author{Vincent Foley-Bourgon (FOLV08078309) \\
Eric Thivierge (THIE09016601)}
\maketitle

\section{Description du travail}

Dans ce travail, nous devions créer les structures de données
nécessaires pour le fonctionnement du système d'exploitation Kaya
ainsi que les différentes opérations sur ces structures.


\section{Décisions techniques}

\subsection{Terminaison d'un processus parent}

Lorsque un processus possédant des enfants est terminé, il existe deux
stratégies pour que les enfants ne soient pas orphelins:

\begin{itemize}
\item Terminer tous les enfants de façon récursive;
\item Ajouter les enfants aux enfants de leur grand-parent.
\end{itemize}

Nous avons choisi d'implanter la première stratégie, car il s'agit de
la stratégie la plus simple à implanter en code: dans la fonction
\emph{removeChild}, on à simplement à faire des appels récursifs
jusqu'à ce que le processus ne possède plus d'enfants.  Cela se fait
en ajouter les deux lignes de code suivantes:

\begin{verbatim}
while (!emptyChild(child))
    removeChild(child);
\end{verbatim}

(On fait de même dans la fonction \emph{outChild}.)


\subsection{Classification des sémaphores}

Beaucoup d'opérations sur les sémaphores dépendent de leur état
actuel.  Il existe 3 états possible pour les sémaphores:

\begin{itemize}
\item Si un sémaphore est libre, sa procQ est vide et elle se trouve
  dans la liste \emph{semdFree};
\item Si un sémaphore vient d'être acquise par \emph{initSemd}, sa
  procQ est vide, mais elle ne se trouve pas dans les listes
  \emph{semdFree} ou \emph{ASL};
\item Si un sémaphore possède des processus dans sa procQ, elle se
  trouve dans la liste \emph{ASL}.
\end{itemize}

Afin de bien distinguer ses 3 états, nous avons introduit dans la
structure \emph{semd\_t} le champ \emph{s\_state} qui contient l'une des
trois valeurs suivantes: {\tt ST\_FREE}, {\tt ST\_ACQUIRED}, ou {\tt
  ST\_ASL}.

Ce champ permet de rapidement (temps constant) obtenir le statut d'un
sémaphore, plutôt que de parcourir les listes \emph{semdFree} et
\emph{ASL} et de vérifier le champ \emph{s\_procQ} (temps linéaire).
De plus, cela exprime bien les trois états dans lequel un sémaphore
peut se trouver, plutôt que d'espérer que le programmeur se rappelle
des états possibles en examinant le code source.

\subsection{Indirection supplémentaire pour initSemD}

Nous utilisons la fonction \emph{initSemD} afin de créer et
d'initialiser un sémaphore dans le code client.  Comme la structure de
\emph{semd\_t} est privée, il est impossible de créer un sémaphore à
l'extérieur du module sema.c (e.g. sa taille est inconnue à la
compilation).  Ainsi, nous avons modifié le prototype de
\emph{initSemD} pour qu'il prenne non pas un pointeur vers un
\emph{semd\_t}, mais un pointeur vers un pointeur vers un
\emph{semd\_t}.  Cette stratégie nous permet de faire pointer une
variable du code client vers un des descripteurs de sémaphore privés à
sema.c.

Voici comment utiliser \emph{initSemD} avec le nouveau prototype.

\begin{verbatim}
semd_t *s;
if (initSemD(&s, 42))
  term_puts("Acquisition de sémaphore réussie.\n");
else
  term_puts("Acquisition de sémaphore échouée.\n");
\end{verbatim}


\section{Stratégies de tests}

Nous avons implanté une batterie complète de tests pour vérifier le
comportement de nos fonctions.  Nous n'avons pas utilisé un
``framework'' déjà existant, et avons opté pour un cadre ultra
minimal.  Une fonction de test possède la signature {\tt void $\to$
  int} où 0 signifie un échec et toute autre valeur signifie une
réussite; cette stratégie ne permet pas de facilement déterminer
quelles assertions dans la fonction de test ont échoué.  Une fonction
nommée \emph{test} possédant la signature {\tt (string * (void $\to$
  int)) $\to$ void} affiche si un test a réussi ou échoué.

Nous avons rajouté l'option {\tt -DDEBUG} au paramètre {\tt CFLAGS} du
Makefile afin que nos fonctions accesseurs soient compilées et
qu'elles puissent être désactivées facilement lorsque ces tests ne
seront plus nécessaires.

\subsection{Blackbox testing}

Pour les fonctions qui le permettent, nous avons utilisé une stratégie
de ``blackbox testing'': nous n'examinons pas l'état interne des
structures de données, mais vérifions uniquement si la sortie d'une
fonction correspond à l'entrée qui lui a été donné\footnote{Nous
  considérons l'environnement global comme étant un des paramètres
  d'entrée.}.  Des fonctions telles que \emph{mkEmptyProcQ},
\emph{emptyProcQ}, \emph{removeProcQ}, \emph{emptyChild} et
\emph{headBlocked} se prêtent à ce genre de tests.

\subsection{Whitebox testing}

Dans les langages de programmation impératifs, certaines fonctions
sont appelées pour leurs effets de bord et ne retourne pas de
résultats utiles.  Par exemple, les fonctions \emph{initProc} et
\emph{initASL} possèdent la signature {\tt void $\to$ void}.  Comme on
ne peut rien vérifier de leur comportement externe, on utilise une
stratégie de ``whitebox testing'' où nous examinons les changements
que ces fonctions ont sur les différentes structures de données du
système.  Cette stratégie de test est plus fragile que le ``blackbox
testing'', car si la structure interne des objets est modifiée durant
le développement du logiciel (par exemple: changement de nom ou de
type d'un champ), ces tests doivent être mis à jour.

Comme le contenu des structures de données de notre programme sont
privées à leur unité de compilation, nous avons ajouté des fonctions
accesseurs qui permettent d'aller chercher ces valeurs.  Ces fonctions
ne sont compilées que si le symbole \emph{DEBUG} existe.

\subsection{Résultats des tests}

Les tests, en conjonction avec le débogueur gdb, ont été cruciaux dans
l'écriture du logiciel.  Chaque fonction possède sa propre suite de
tests, et les tests étaient exécutés constamment durant le
développement.  Lorsque une erreur était rencontrée, que ce soit un
résultat incorrect ou une erreur de programmation (ex.: segmentation
fault), il était facile et rapide de trouver l'endroit et la cause de
l'échec du test.

\section{Références}


\begin{itemize}
\item Michael Goldweber, Renzo Davoli. Student Guide to the Kaya
  Operating System. Lulu Books, 2009
  (\url{http://www.cs.unibo.it/~renzo/so/Kaya.pdf})
\end{itemize}


\end{document}
