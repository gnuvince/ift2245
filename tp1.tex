\documentclass{article}


\usepackage[utf8]{inputenc}
\usepackage{amsfonts}            %For \leadsto
\usepackage{amsmath}             %For \text
\usepackage{fancybox}            %For \ovalbox
\usepackage[letterpaper,margin=1in,includehead]{geometry}

\title{Travail pratique \#1}
\author{IFT-2245}

\begin{document}

\maketitle

{\centering \ovalbox{\large ¡¡ Dû le 14 février à midi !!} \\}

\newcommand \uMPS {\ensuremath\mu\textsl{MPS2}}
\newcommand \kw [1] {\textsf{#1}}
\newcommand \id [1] {\texttt{#1}}
\newcommand \punc [1] {\kw{`#1'}}
\newenvironment{outitemize}{
  \begin{itemize}
  \let \origitem \item \def \item {\origitem[]\hspace{-18pt}}
}{
  \end{itemize}
}

\section{Survol}

Ce TP a pour but de vous introduire au codage C sur le simulateur \uMPS, et
de commencer le codage de votre système d'exploitation \emph{Kaya}.

Les étapes sont les suivantes:
\begin{enumerate}
\item Lire et comprendre cette donnée.
\item Écrire le code, le tester, le choyer, le peaufiner, ...
\item Écrire un rapport.  Il doit décrire \textbf{votre} expérience pendant
  les points précédents: problèmes rencontrés, surprises, choix que vous
  avez dû faire, options que vous avez sciemment rejetées, etc...  Le
  rapport ne doit pas excéder 4 pages.
\end{enumerate}

Ce travail est à faire en groupes de 2 étudiants.  Le rapport doit être
écrit en \LaTeX\ (compilable sur \texttt{frontal.iro}).  Le rapport et le code
sont à remettre par remise électronique avant la date indiquée.
Aucun retard ne sera accepté.  Indiquez clairement vos noms au début de
chaque fichier.

Si un étudiant préfère travailler seul, libre à lui, mais l'évaluation de
son travail n'en tiendra pas compte.  Si un étudiant ne trouve pas de
partenaire, il doit me contacter au plus tard jeudi 2 février
\textbf{avant} le cours.  Des groupes de 3 ou plus sont \textbf{exclus}.

\section{Contexte}

Dans ce cours vous allez implanter votre propre système d'exploitation,
dénomé Kaya, écrit en C et qui roule sur l'émulateur d'architecture MIPS
\uMPS.  Dans ce premier TP vous allez seulement écrire quelques routines
de gestion de structures de données simples.

Plus spécifiquement il faut implanter des routines de manipulation d'objets
qui joueront le rôle (dans les TPs à venir) de processus et de sémaphores.
Comme un noyau n'a pas accès aux facilités de la librairies standard de
C (e.g.~\texttt{printf}, \texttt{malloc}, ...) vu que cette librairie
a elle-même besoin d'un noyau pour lui fournir les fonctionnalités
correspondantes, il faudra gérer la mémoire de manière statique: pour les
processus, vous utiliserez une table préallouée de processus qui limitera
arbitrairement le nombre maximum de processus que votre système pourra
supporter à un moment donné; pour les sémaphores vous laisserez les
utilisateurs des routines faire leur propre allocation.

\section{Fichiers fournis}

4 fichiers sont fournis qui déclarent les fonctions que vous devez implanter
ainsi que les types des objets que vous devez utiliser pour implanter
ces fonctions:
\begin{itemize}
\item \texttt{proc.h}: déclarations de fonctions de manipulation
  de processus.
\item \texttt{proc.c}: type de ces processus et implantation (à remplir)
  des fonctions déclarées.
\item \texttt{sema.h}: déclarations de fonctions de manipulation
  de sémaphores.
\item \texttt{sema.c}: type de ces sémaphores et implantation (à remplir)
  des fonctions déclarées.
\item \texttt{tp1.tex}: le code \LaTeX\ du TP, au cas où vous voudriez vous
  en inspirer pour votre rapport.
\end{itemize}

Il y a aussi un document nommé \emph{Student Guide to the Kaya Operating
  System} qui décrit une version antérieure de Kaya, disponible
à \texttt{http://mps.sourceforge.net/documentation.html}, ainsi que le
document \emph{µMPS2 Principles of Operation}, disponible à la même adresse,
qui décrit la machine simulée par \uMPS.

\section{À faire}

Il vous faut faire les choses suivantes:
\begin{itemize}
\item Implanter les fonctions demandées dans les deux fichiers
  \texttt{.c} fournis.
\item Écrire un fichier \texttt{tp1test.c} qui fait divers tests (libre
  à vous) de ces fonctions en affichant sur le terminal les succès/échecs de
  ces tests.
\item Écrire un fichier \texttt{Makefile} pour compiler tout ces fichiers et
  en générer un fichier \texttt{kernel.core.umps} que l'on peut passer
  à \uMPS.
\item Écrire le fichier \texttt{rapport-tp1.tex}.
\end{itemize}
Pour \texttt{tp1test.c} et \texttt{Makefile}, je vous recommande de vous
inspirer des fichiers de l'exemple \texttt{hello-umps}.

\section{Remise}

Il faudra remettre électroniquement, via la page Moodle (aussi connu sous le
nom de StudiUM) du cours, 1 fichier au format \texttt{tar.gz} contenant
7 fichiers (les 4 fichier C fournis, plus les 3 nouveaux).
Indiquez clairement vos noms au début de chaque fichier.

\section{Barème}

Les points en jeux seront divisés comme suit:
\begin{itemize}
\item 50\% points pour le codages des fonctions.
\item 20\% pour le codage du Makefile et du code de tests.
\item 10\% pour le choix des tests.
\item 20\% pour le rapport.
\end{itemize}
Les points sur le code seront pour moitié basés sur de simple tests pour
vérifier le comportement de vos programmes et pour moitié basé sur une revue
manuelle de code.  Les critères par ordre d'importance:
\begin{itemize}
\item Le code doit être correct dans tous les cas, même (surtout) les
  plus tordus.
\item Le code ne doit pas être excessivement lent.
\item Coder de manière défensive: faites aussi peu d'hypothèses que
  possible; exposer le code aussi peu que possible (par exemple, utiliser
  \texttt{static} chaque fois que possible et garder les déclarations de
  types dans les fichiers \texttt{.c} plutôt que \texttt{.h} lorsque c'est
  possible).
\item Le code doit suivre les conventions stylistiques habituelles:
  indentation standard, ne jamais dépasser 80 colonnes, les commentaires
  sont traités avec dignité (e.g.~ils ont droit à une majuscule au début et
  une ponctuation à la fin), ...
\item Si une partie du comportement n'est pas spécifiée dans la donnée, il
  faut le mentionner dans le code et/ou le rapport et expliquer quel choix
  vous avez fait et pourquoi.
\item Le code doit être \emph{bien} commenté, i.e.~ni trop, ni trop peu;
  expliquer ce que le code fait (et ne fait pas), sans le paraphraser.
\item \emph{Faute avouée est à moitié pardonnée}: si votre code s'écarte du
  comportement stipulé par la donnée, il est préférable de le mentionner
  et expliquer dans le rapport, pour bien montrer que vous en êtes au
  moins conscient.
\end{itemize}

\end{document}
